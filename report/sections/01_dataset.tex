% Section 1: Dataset
% Describe the dataset chosen and why it is interesting or useful to analyze.
% How did you collect the data, choose the categories, and split it into documents?
% Include tables and figures to show the size of your dataset.

\section{Dataset}

% TODO: Fill in with actual dataset information
% This section should include:
% - Dataset description and source (LogHub)
% - Why it's interesting/useful
% - How data was collected
% - How categories were chosen
% - How documents were split
% - Tables showing:
%   - Number of documents per category
%   - Average number of tokens per document per category
%   - Total corpus statistics

\textit{This section will be populated after running the corpus analysis pipeline.}

Example structure:
\begin{itemize}
    \item \textbf{Dataset Source}: LogHub repository
    \item \textbf{Categories}: [To be filled]
    \item \textbf{Document Count}: [To be filled]
    \item \textbf{Collection Method}: [To be filled]
\end{itemize}

% Table example (to be filled with actual data):
% \begin{table}[H]
% \centering
% \begin{tabular}{lcc}
% \toprule
% Category & Documents & Avg Tokens/Doc \\
% \midrule
% Category 1 & XXX & XXX \\
% Category 2 & XXX & XXX \\
% \bottomrule
% \end{tabular}
% \caption{Dataset statistics by category}
% \end{table}
